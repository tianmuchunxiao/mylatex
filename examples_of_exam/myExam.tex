\documentclass[11pt,addpoints,answers]{exam}

\usepackage{ctex}

\usepackage{amsmath,amssymb}

\usepackage{color}

\usepackage{geometry}
\geometry{paperwidth=420mm,paperheight=297mm,top=20mm,left=20mm,right=20mm,bottom=20mm}

\usepackage{multicol}
\setlength{\columnsep}{1cm}
\setlength{\columnseprule}{1pt}

\linespread{1.5}

\pagestyle{headandfoot}
\headrule
\header{溧阳市戴埠高级中学}{高三数学}{\today}
\footer{}{共\numpages 页 第\thepage 页}{}

\usepackage{zhnumber}

\renewcommand{\thesection}{\zhnum{section}、}
\renewcommand{\thepartno}{\arabic{partno}}
\renewcommand\subpartlabel{(\thesubpart)}
\settowidth{\leftmargin}{-2.5em}



\usepackage{titlesec}
\titleformat*{\section}{\heiti\zihao{5}}


\pointpoints{分}{分}

\begin{document}

\CorrectChoiceEmphasis{\color{red}}

\begin{multicols}{3}

    {\centering \zihao{-2}{\kaishu 2022年普通高等学校招生全国统一考试}

        \centering \zihao{2}{\kaishu 数学}

        \centering \zihao{5}{班级\fillin\qquad  姓名\fillin\qquad  学号\fillin  }

    }
    \zihao{5}

    \begin{questions}
        \section{选择题:本题共8小题,每小题5分,共40分.在每小题给出的四个选项中,只有一项是符合题目要求的.}

        \question
        若集合$M=\left\{ x\left|\sqrt{x}<4\right. \right\}$, $N=\left\{x\left|3x \geqslant 1\right.\right\}$,则$M\cap N=$

        \begin{oneparchoices}
            \choice $\left\{ x\left|0\leqslant x<2\right. \right\}$
            \choice $\left\{ x\left|\dfrac{1}{3}\leqslant x<2\right. \right\}$
            \choice $\left\{ x\left|3\leqslant x<16\right. \right\}$
            \correctchoice $\left\{ x\left|\dfrac{1}{3}\leqslant x<16\right. \right\}$
        \end{oneparchoices}


        \question
        若$\mathrm{i} \left( 1 - z \right)=1$,则$z+\bar{z}=$

        \begin{oneparchoices}
            \choice $-2$
            \choice $-1$
            \choice $1$
            \choice $2$
        \end{oneparchoices}

        \question
        在$\triangle ABC$中,点$D$在边$AB$上,$BD=2DA$.记$\overrightarrow{CA}=\boldsymbol{m}$,$\overrightarrow{CD}=\boldsymbol{n}$,则$\overrightarrow{CB}=$

        \begin{oneparchoices}
            \choice $3\boldsymbol{m}-2\boldsymbol{n}$
            \choice $-2\boldsymbol{m}+3\boldsymbol{n}$
            \choice $3\boldsymbol{m}+2\boldsymbol{n}$
            \choice $2\boldsymbol{m}+3\boldsymbol{n}$
        \end{oneparchoices}

        \question
        南水北调工程缓解了北方一些地区水郑源短缺问题,其中一部分水蓄入某水库.已知该水库水位为海拔$148.5m$时,相应水面的面积为$140.0km^2$;水位为海拔$157.5m$时,相应水面的面积为$180.0km^2$.将该水库在这两个水位间的形状看作一棱台,则该水库水位从海拔$148.5m$上升到海拔$157.5m$时增加的水量为($\sqrt{7}\approx 2.65$)

        \begin{oneparchoices}
            \choice $1.0\times 10^9m^3$
            \choice $1.2\times 10^9m^3$
            \choice $1.4\times 10^9m^3$
            \choice $1.6\times 10^9m^3$
        \end{oneparchoices}

        \question
        从2至8的7个整数中随机取2个不同的数,则这2个数互质的概率为

        \begin{oneparchoices}
            \choice $\dfrac{1}{6}$
            \choice $\dfrac{1}{3}$
            \choice $\dfrac{1}{2}$
            \choice $\dfrac{2}{3}$
        \end{oneparchoices}

        \question
        记函数$f\left(x\right)=\sin \left(\omega x+\dfrac{\pi}{4}\right)+b\left(\omega>0\right)$的最小正周期为$T$.若$\dfrac{2\pi}{3}<T<\pi$,且$y=f\left(x\right)$的图像关于点$\left(\dfrac{3\pi}{2},2\right)$中心对称,则$f\left(\dfrac{\pi}{2}\right)=$

        \begin{oneparchoices}
            \choice $1$
            \choice $\dfrac{3}{2}$
            \choice $\dfrac{5}{2}$
            \choice $3$
        \end{oneparchoices}

        \question
        设$a=0.1\mathrm{e}^{0.1}$,$b=\dfrac{1}{9}$,$c=-\ln 0.9$,则

        \begin{oneparchoices}
            \choice $a<b<c$
            \choice $c<b<a$
            \choice $c<a<b$
            \choice $a<c<b$
        \end{oneparchoices}

        \question
        已知正四棱锥的侧枝长为$l$,其各顶点都在同一球面上.若该球的体积为$36\pi$,且$3\leqslant l\leqslant3\sqrt{3}$,则该正四棱锥体积取值范围是

        \begin{oneparchoices}
            \choice $\left[8,\dfrac{81}{4}\right]$
            \choice $\left[\dfrac{27}{4},\dfrac{81}{4}\right]$
            \choice $\left[\dfrac{27}{4},\dfrac{64}{3}\right]$
            \choice $\left[18,27\right]$
        \end{oneparchoices}

        \section{选择题:本题共4小题,每小题5分,共20分.在每小题给出的选项中,有多项符合题目要求.全部选对的得5分,部分选对的得2分,有选错的得0分.}

        \question
        已知正方体$ABCD-A_1B_1C_1D_1$,则

        \begin{choices}
            \choice 直线$BC_1$与$DA_1$所成的角为$90^{\circ}$
            \choice 直线$BC_1$与$CA_1$所成的角为$90^{\circ}$
            \choice 直线$BC_1$与平面$BB_1D_1D$所成的角为$45^{\circ}$
            \choice 直线$BC_1$与平面$ABCD$所成的角为$45^{\circ}$
        \end{choices}

        \question
        已知函数$f\left(x\right)=x^3-x+1$,则

        \begin{choices}
            \choice $f\left(x\right)$有两个极值点
            \choice $\left(x\right)$有三个零点
            \choice 点$\left(0,1\right)$是曲线$y=f\left(x\right)$的对中心
            \choice 直线$y=2x$是曲线$y=f\left(x\right)$的切线
        \end{choices}

        \question
        已知$O$为坐标原点,点$A\left(1,1\right)$在抛物线$C$:$x^2=2py\left(p>0\right)$上,过点$B\left(0,-1\right)$的直线交$C$于$P$,$Q$两点,则

        \begin{choices}
            \choice $C$的准线为$y=-1$
            \choice 直线$AB$与$C$相切
            \choice $\left|OP\right|\cdot\left|OQ\right|>\left|OA\right|^{2}$
            \choice $\left|BP\right|\cdot\left|BQ\right|>\left|BA\right|^{2}$
        \end{choices}

        \question
        已知函数$f\left(x\right)$及其导函数$f^{\prime}\left(x\right)$的定义域均为$\mathbf{R}$,记$g\left(x\right)=f^{\prime}\left(x\right)$.若$f\left(\dfrac{3}{2}-2x\right)$,$g\left(2+x\right)$均为偶函数,则

        \begin{choices}
            \choice $f\left(0\right)=0$
            \choice $g\left(-\dfrac{1}{2}\right)=0$
            \choice $f\left(-1\right)=f\left(4\right)$
            \choice $g\left(-1\right)=g\left(2\right)$
        \end{choices}

        \section{填空题:本题共4小题,每小题5分,共20分.}

        \question
        $\left(1-\dfrac{y}{x}\right)\left(x+y\right)^{8}$的展开式中$x^{2}y^{6}$的系数为\fillin (用数字作答).

        \question
        写出与圆$x^2+y^2=1$和$\left(x-3\right)^2+\left(y-4\right)^2=16$都相切的条直线的方程\fillin .

        \question
        若曲线$y=\left(x+a\right)\mathrm{e}^x$有两条过坐标原点的切线,则$a$的取值范围是\fillin .

        \question
        已知椭圆$C$:$\dfrac{x^2}{a^2}+\dfrac{y^2}{b^2}=1\left(a>b>0\right)$,$C$的上顶点为$A$,两个焦点为$F_1$,$F_2$,离心率为$\dfrac{1}{2}$.过$F_1$且垂直于$AF_2$的直线与$C$交于$D$,$E$两点,$\left|DE\right|=6$,则$\triangle ADE$的周长是\fillin .


        \section{解答题:本题共6小题,共70分.解答应写出文字说明、证明过程或演算步骤.}

        \question[10]
        记$S_n$为数列$\left\{a_n\right\}$的前$n$项和,已知$a_1=1$,$\left\{\dfrac{S_n}{a_n}\right\}$是公差为$\dfrac{1}{3}$的等差数列.
        \begin{parts}
            \part 求$\left\{a_n\right\}$的通项公式;
            \part 证明:$\dfrac{1}{a_1}+\dfrac{1}{a_2}+\cdots+\dfrac{1}{a_n}<2$.
        \end{parts}
        \fillwithdottedlines{3.1in}

        \question[12]
        记$\triangle ABC$的内角$A$,$B$,$C$的对边分别为$a$,$b$,$c$,已知$\dfrac{\cos A}{1+\sin A}=\dfrac{\sin 2B}{1+\cos 2B}$.
        \begin{parts}
            \part 若$C=\dfrac{2\pi}{3}$,求$B$;
            \part 求$\dfrac{a^2+b^2}{c^2}$的最小值.
        \end{parts}
        \fillwithdottedlines{3.1in}

        \question[12]
        如图,直三棱柱$ABC-A_1B_1C_1$的体积为$4$,$\triangle A_1BC$的面积为$2\sqrt{2}$.
        \begin{parts}
            \part 求$A$到平面$A_1BC$的距离;
            \part 设$D$为$A_1C$的中点,$AA_1=AB$,平面$A_1BC\bot$平面$ABB_1A_1$,求二面角$A-BD-C$的正弦值.
        \end{parts}
        \fillwithdottedlines{4in}

        \question[12]
        一医疗为研究某地的一种地方性疾病与当地居民的卫生习惯卫生习惯分为良好和不够良好两类)的关系,在已患该疾病的病例中随机调查了100例(称为病例组),同时在未患该疾病的人群中随机调查了100人(称为对照组),得到如下数据:
        \begin{center}
            \begin{tabular}{|c|c|c|}
                \hline
                       & 不够良好 & 良好 \\
                \hline
                病例组 & 40       & 60   \\
                \hline
                对照组 & 10       & 90   \\
                \hline
            \end{tabular}
        \end{center}
        \begin{parts}
            \part 能否有$99\%$的把握认为患该疾病群体与未患该疾病群体的卫生习惯有差异?
            \part 从该地的人群中任选一人,$A$表示事件“选到的人卫生习惯不够良好”,$B$表示事件“选到的人患有该疾病”,$\dfrac{P\left(A | B\right)}{P\left(\bar{A}|B\right)}$与$\dfrac{P\left(B | \bar{A}\right)}{P\left(\bar{B}|\bar{A}\right)}$的比值是卫生习惯不够良好对患该疾病风险程度的一项度量指标,记该指标为$R$.
            \begin{subparts}
                \subpart 证明:$R=\dfrac{P\left(A | B\right)}{P\left(\bar{A}|B\right)} \cdot \dfrac{P\left(\bar{A} | \bar{B}\right)}{P\left(A|\bar{B}\right)}$;
                \subpart 利用该调查数据,给出$P\left(A|B\right)$,$P\left(A|\bar{B}\right)$的估计值,并利用(i)的结果给出$R$的估计值.
            \end{subparts}
        \end{parts}
        附:$K^2=\dfrac{n\left(ad-bc\right)^2}{\left(a+b\right)\left(c+d\right)\left(a+c\right)\left(b+d\right)}$,\[\begin{array}{c|ccc}
                P\left(K^2 \geqslant k\right) & 0.050 & 0.010 & 0.001  \\
                \hline
                k                             & 3.841 & 6.635 & 10.828 \\
            \end{array}\]
        \fillwithdottedlines{3in}
        
        \question[12]
        已知点$A\left(2,1\right)$在双曲线$\dfrac{x^2}{a^2}-\dfrac{y^2}{a^2-1}=1\left(a>1\right)$,直线$l$交$C$于$P$,$Q$两点,直线$AP$,$AQ$的斜率之和为0.

        \begin{parts}
            \part 求$l$的斜率;
            \part 若$\tan \angle PAQ=2\sqrt{2}$,求$\triangle PAQ$的面积.
        \end{parts}
        \fillwithdottedlines{4in}

        \question[12]
        已知函数$f\left(x\right)=\mathrm{e}^x-ax$和$g\left(x\right)=ax-\ln x$有相同的最小值.

        \begin{parts}
            \part 求$a$;
            \part 证明:存在直线$y=b$,其与两条曲线$y=f\left(x\right)$和$y=g\left(x\right)$共有三个不同的交点,且从左到右的三个交点的横坐标成等差数列.
        \end{parts}
        \fillwithdottedlines{8in}

    \end{questions}


\end{multicols}

\end{document}