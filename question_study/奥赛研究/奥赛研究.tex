\documentclass[a4paper,10pt]{ctexbook}

\title{试题研究}
\author{天目春辉}
\date{\today}


\usepackage{geometry}
\geometry{left=15mm,right=15mm,bottom=15mm,top=15mm}

\usepackage{amsmath,amssymb}
\usepackage[dvipdfmx]{graphicx}
\usepackage{tikz}
\usepackage{picinpar}

\usepackage{enumitem}
\usepackage{color}
\usepackage{fancyhdr}
\pagestyle{fancy}
\lhead{Power数学工作室}
\chead{奥赛研究}
\rhead{\today}


\usepackage{tcolorbox}
\tcbuselibrary{theorems}
\tcbuselibrary{breakable}
\tcbuselibrary{skins}
\tcbuselibrary{xparse}
\tcbuselibrary{raster}




\NewTColorBox[auto counter,number within=section]{exercise}{+!O{}}{enhanced,colframe=green!20!black,colback=yellow!10!white,coltitle=green!40!black,fonttitle=\bfseries,underlay={\begin{tcbclipinterior}\shade[inner color=green!80!yellow,outer color=yellow!10!white](interior.north west) circle (2cm);\draw[help lines,step=5mm,yellow!80!black,shift={(interior.north west)}]
(interior.south west) grid (interior.north east);\end{tcbclipinterior}},
title={习题~\thetcbcounter:},
label={exercise@\thetcbcounter},
attach title to upper=\quad,
after upper={\par\hfill\textcolor{green!40!black}{\itshape 解答在~\pageref{solution@\thetcbcounter} 页}},
lowerbox=ignored,
savelowerto=solutions/exercise-\thetcbcounter.tex,
record={\string\solution{\thetcbcounter}{solutions/exercise-\thetcbcounter.tex}},#1}

\NewTotalTColorBox{\solution}{mm}{%
enhanced,colframe=red!20!black,colback=yellow!10!white,coltitle=red!40!black,
fonttitle=\bfseries,
underlay={\begin{tcbclipinterior}
\shade[inner color=red!50!yellow,outer color=yellow!10!white]
(interior.north west) circle (2cm);
\draw[help lines,step=5mm,yellow!80!black,shift={(interior.north west)}]
(interior.south west) grid (interior.north east);
\end{tcbclipinterior}},
title={习题~\ref{exercise@#1} (~\pageref{exercise@#1})的解答:},
phantomlabel={solution@#1},
attach title to upper=\par,
}{\input{#2}}
\tcbset{no solution/.style={no recording,after upper=}}

\newtcbtheorem{question}{例题~}%
  {enhanced, breakable,
    colback = white, colframe = cyan, colbacktitle = cyan,
    attach boxed title to top left = {yshift = -2mm, xshift = 5mm},
    boxed title style = {sharp corners},
    fonttitle = \sffamily\bfseries, separator sign = {.~}}{qst}
\newtcbtheorem{theorem}{定理~}%
  {enhanced, breakable,
    colback = white, colframe = green, colbacktitle = green,
    attach boxed title to top left = {yshift = -2mm, xshift = 5mm},
    boxed title style = {sharp corners},
    fonttitle = \sffamily\bfseries, separator sign = {.~}}{qst}
\newtcbtheorem{property}{性质~}%
  {enhanced, breakable,
    colback = white, colframe = purple, colbacktitle = purple,
    attach boxed title to top left = {yshift = -2mm, xshift = 5mm},
    boxed title style = {sharp corners},
    fonttitle = \sffamily\bfseries, separator sign = {.~}}{qst}
 \newtcbtheorem{point}{知识点~}%
    {enhanced, breakable,
      colback = white, colframe = brown, colbacktitle = brown,
      attach boxed title to top left = {yshift = -2mm, xshift = 5mm},
      boxed title style = {sharp corners},
      fonttitle = \sffamily\bfseries, separator sign = {.~}}{qst}

\begin{document}
\maketitle
\chapter{一般圆锥曲线的性质及应用}


\begin{point}{}{example}
    对于一般圆锥曲线$f\left(x,y\right)=Ax^2+Bxy+Cy^2+Dx+Ey+F=0$,令$\Delta=B^2-4AC$.
    \begin{enumerate}
        \item 若$\Delta<0$,则$f\left(x,y\right)=0$为椭圆型;
        \item 若$\Delta=0$,则$f\left(x,y\right)=0$为抛物线型;
        \item 若$\Delta>0$,则$f\left(x,y\right)=0$为双曲线型.
    \end{enumerate}
\end{point}

\begin{property}{}{example}
    直线$y=kx+m$与圆锥曲线$f\left(x,y\right)=0$若交于两个不同的点$A,B$,则称线段$AB$为该曲线的一条弦,弦长由下载公式给出:
    $$\left|AB\right|=\dfrac{1}{\left|a\right|}\sqrt{\left(1+k^2\right)\left(b^2-4ac\right)}.$$
    其中$a,b,c$为$y=kx+m$代入$f\left(x,y\right)=0$中消去$y$所得二次方程$ax^2+bx+c=0$的各次项系数.
\end{property}

\begin{property}{}{example}
    设$AB$是圆锥曲线过焦点$F$的弦,其长度记作$l$,$AB$相对于焦点所在对称轴的倾角为$\theta\left(\theta\neq 90^{\circ}\right)$,$\tan \theta=k$,$e$为离心率,$p$为焦点到相应准线的距离,则有$l$与$k$的关系:

    $$l=\dfrac{2ep\left(1+k^2\right)}{\left(1+k^2\right)-e^2}\text{或}k^2=\dfrac{e^2l}{l-2ep}-1.$$
    \tcbsubtitle[]{证明}
    由圆锥曲线统一的极坐标方程$\varrho=\dfrac{ep}{1-e\cos\theta}$,得$\left|AF\right|=\dfrac{ep}{1-e\cos\theta}$,$\left|BF\right|=\dfrac{ep}{1+e\cos\theta}$,从而$l=\left|AF\right|+\left|BF\right|=\dfrac{2ep\left(1+k^2\right)}{\left(1+k^2\right)-e^2}$.

    再注意到$\cos^2 \theta=\dfrac{1}{1+\tan^2 \theta}=\dfrac{1}{1+k^2}$,代入即证得.

    \heiti{注}
    \begin{enumerate}
        \item $\theta =90^{\circ}$时,$l=\dfrac{2ep\left(1+k^2\right)}{\left(1+k^2\right)-e^2}=2ep$通径长);
        \item 对于椭圆和双曲线$p=\left|\dfrac{a^2}{c}-c\right|=\dfrac{b^2}{c}$.
    \end{enumerate}
\end{property}

\begin{property}{}{example}
    设$F$为圆锥曲线焦点,其相应准线为$L$,作一直线交圆锥曲线于$A,B$,交$L$于$M$,则$FM$平分$\triangle AFB$的$\angle AFB$的外角(即$BF,AF$与直线$MF$成等角).
    \tcbsubtitle[]{证明}

    如图,从$A,B$分别向$L$作垂线$AA^{\prime}$与$BB^{\prime}$,垂足为$A^{\prime},B^{\prime}$,由圆锥曲线定义,有

    \begin{tikzpicture}
        \draw(-2,0)--(6,0) (0,-3)--(0,3);
        \draw[cyan,domain=30:330,smooth] plot(3 / cos(\x r),2*tan(\x r)) node at (2.3,0.6){$y=0.3sin(4x)$};
    \end{tikzpicture}
\end{property}



\begin{question}{函数}{example}
    已知函数 $ f(x) = (x - 2)\mathrm{e}^{2} + a (x - 1)^{2} $ 有两个零点.
    \begin{enumerate}[label=(\arabic*)]
        \item 求 $ a $ 的取值范围;
        \item 设 $ x_{1} $, $ x_{2} $ 是 $ f(x) $ 的两个零点,证明 $ x_{1} + x_{2} < 2 $.
    \end{enumerate}
    \tcbsubtitle[]{解答}
    这是一个解答
\end{question}

\tcbstartrecording\relax
\begin{exercise}
    Compute the derivative of the following function:
    \begin{equation*}
        f(x)=\sin((\sin x)^2)
    \end{equation*}

    \tcblower

    The derivative is:
    \begin{align*}
        f'(x) & = \left( \sin((\sin x)^2) \right)'
        =\cos((\sin x)^2) 2\sin x \cos x.
    \end{align*}
\end{exercise}

\begin{exercise}[no solution]
    It holds:
    \begin{equation*}
        \frac{d}{dx}\left(\ln|x|\right) = \frac{1}{x}.
    \end{equation*}
\end{exercise}


\tcbstoprecording

\tcbinputrecords

\end{document}